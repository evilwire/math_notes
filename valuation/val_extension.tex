\section{Extensions of Valuations}

We begin the discussion by talking about extending ring and field 
morphisms, which will be lay the ground work for establishing the 
following result:

\begin{thm}\label{thm_ext_val}
Let $K$ be a subfield of a field $L$. Then a valuation on $K$ has 
an extension to a valuation on $L$.
\end{thm}

The proof of this theorem will be given at the end of the section.
Let us begin with the following observation:

\begin{prop}[\cite{LangAlg} V\S2.8]\label{prop_lang_2_8}
Let $k$ be a field, $E$ an algebraic extension of $k$, and 
$\sigma: k \to L$ a field homomorphism from $k$ to an 
algebraically closed field $L$. Then there exists an extension of 
$\sigma$ to a homomorphism $\sigma': E \to L$.
\end{prop}

\begin{proof}
In the case where $E$ is a finite extension, then this is clear.
For an arbitrary extension, let $\Class{F}$ denote the collection
of all pairs $(F, \tau)$ where $F$ is a subfield of $E$ and 
$\tau: F \to L$ is a morphism for which $\tau|_k = \sigma$. 
Order $\Class{F}$ by $(F, \tau) \leq (F', \tau')$ if $F$ is a 
subfield of $F'$ and $\tau'|_F = \tau$.

Apply Zorn's lemma, and let $(K, \lambda)$ be a maximal element
in $\Class{F}$. We show that $\lambda$ is the desired extension to 
$E$. In particular and sufficiently, $K = E$. This follows 
trivially, since otherwise, fix $x \in E$, $x \notin K$, and 
notice that $\lambda$ extends to $\lambda': K(x) \to L$, 
contradicting the maximality of $(K, \lambda)$.
\end{proof}

In particular:

\begin{cor}
A field homomorphism $E \to L$ where $L$ is algebraically closed
can be extended to the algebraic closure of $E$.
\end{cor}

In fact, we have something stronger.

\begin{prop}
Let $E$ be a subfield of $F$, and $L$ is any algebraically closed
field with transcendence degree of $L$ over $E$ at least as large 
as the transcendence degree of $F$ over $E$. Then any $\phi: E \to 
L$ can be extended to $F \to L$.
\end{prop}
\begin{proof}
Factor $E \to F$ into an algebraic extension followed by a 
transcendental extension. By replacing $E$ by its algebraic 
closure in $F$ via Prop. \ref{prop_lang_2_8}, we may assume that 
$E \to F$ is transcendental.

At this point, any injective maps on the transcendental generators 
of $F$ over $E$ to the transcendental generators of $L$ over $E$
will induce an extension of $\phi$ to a map $E \to L$.
\end{proof}

The following two propositions can be seen as the ring-theoretic
counterparts of the preceding proposition:

\begin{prop}\label{prop_lang_3_1}
Let $A$ be a subring of $B$ which is integral over $A$. Let
$\phi: A \to L$ be a homomorphism into an algebraically closed 
field $L$. Then $\phi$ has an extension to a homomorphism of $B$
into $L$.
\end{prop}

\begin{proof}
We first reduce the case to local rings with the following lemma.

\begin{lem}\label{lem_lang_3_1}
Let $\phi : A \to L$ be a ring morphism with $L$ an algebraically
closed field, and fix $\ideal{p}$ any prime ideal containing the
kernel. Then there exists an extension of $\phi$ to the 
$A_{\ideal{p}}$.
\end{lem}
\begin{proof}[Proof of Lemma.]
It suffices to show that the image of $S = A - \ideal{p}$ consist
entirely of invertible elements. This follows from the fact that
$\ker \phi$ is contained in $\ideal{p}$. The existence of the
extension follows from the universal property of $A_{\ideal{p}}$.
\end{proof}

Now, let $\ideal{p} = \ker \phi$. By replacing $A$ by $A_{\ideal{p}}$
and $B$ by $B_{\ideal{p}B}$, we may assume that $A$ and $B$ are
local, and the maximal ideal of $A$ is precisely the kernel of $\phi$,
and factors through the residue field $A/\ideal{p}$.

Furthermore, since $B$ is integral over $A$, there exists a maximal
ideal $\ideal{m}$ such that $\ideal{p} = \ideal{m}$ (see 
\cite{MatsCA}, \S9 Lem. 2), and that $A/\ideal{p} \to B/\ideal{m}$ is
a finite field extension. The proposition now follows from the
commutativity of the diagram:
\[
\begin{tikzcd}
B \arrow{r} \arrow{d} 
& B/\ideal{m} \arrow{d} \arrow[dotted]{rd} \\
A \arrow{r} 
& A/\ideal{p} \arrow{r}
& L
\end{tikzcd}
\]
\end{proof}

The following begin to suggest a valuative behavior, and inspire 
our discussion of places.

\begin{prop}\label{thm_lang_3_2}
Let $A$ be a subring of a field $K$ which is in turn a subfield 
of an algebraically closed field $L$, and let $x \in K$ be a 
nonzero element. Then any morphism $\phi: A \to L$ emits an 
extension to a homomorphism $A[x] \to L$ or an extension 
$A[x^{-1}] \to L$. (In other words, if there does not exist an 
extension $A[x] \to L$ then, there surely must exist an extension 
$A[x^{-1}] \to L$.)
\end{prop}
\begin{proof}
By Lemma \ref{lem_lang_3_1}, we may assume that $A$ is local, and
that the maximal $\ideal{m}$ is the kernel of $\phi: A \to L$.
There are two cases: $\ideal{m}A[x^{-1}] = A[x^{-1}]$ or 
$\ideal{m}A[x^{-1}] \neq A[x^{-1}]$, and we will see that the former
leads to an extension $A[x] \to L$ and the latter an extension of
$A[x^{-1}] \to L$.

\pfitem{Case $\ideal{m}A[x^{-1}] = A[x^{-1}]$}: we can write
\[
1 = a_0 + a_1x^{-1} + \cdots + a_nx^{n}
\]
with $a_i \in \ideal{m}, a_n \neq 0$. Dividing by $1 - a_0$, we 
may assume that $a_0 = 0$. Multiplying by $x^n$, we see that $x$ 
is integral over $A$. The extension $A[x] \to L$ follows from 
Prop. \label{prop_lang_3_1}.

\pfitem{Case $\ideal{m}A[x^{-1}] \neq A[x^{-1}]$}: we may assume 
that $\ideal{m}A[x^{-1}] \subset \ideal{m'}$ for some maximal 
ideal $\ideal{m'}$ of $A[x^{-1}]$. Furthermore, since $1 \notin 
\ideal{m'}$ and $\ideal{m} \subset A \cap \ideal{m'}$, $A \cap 
\ideal{m'} = \ideal{m}$ by the maximality of $\ideal{m}$.

Furthermore, $\phi$ factors through $A/\ideal{m}$.
\end{proof}
