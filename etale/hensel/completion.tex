\section{Filtrations, completions, and Artin-Rees.}

We begin the discussion with a topological excursion, as to lay 
the ground work for discussions regarding ``adic'' completion in 
this section. This topic is developed from the view point of a 
topological group. In this case, the group is a commutative ring 
$R$ with a topology given by a system of ideals; certainly, the 
ideas are more general. We do not treat the topic in full 
generality, though the ideas will come up again in our discussion 
of profinite groups in Appendix \ref{appendix:profinite}. The
results here are taken from \cite{AM}, \cite{MatsCA} and 
\cite{Eisenbud} as quasi-fascimile. Readers who are familiar with
the ``adic'' completion and related results can safely skip this
section.

Throughout, let $G$ be a topological group.\footnote{There are two 
equivalent ways of understanding this notion. First --- the more 
classical way --- $G$ is a group together with a topology on its 
underlying set in which left/right multiplication with respect to 
any element $g \in G$ and taking inverse are continuous are 
continuous functions from $G$ to $G$.  Second --- the more hipster 
way --- is to say that $G$ is a group object in the category of 
topological spaces. That is, there are maps of topological space
\[
\mu: G \times G \to G\;\;\;\textrm{(composition) and}\;\;\;\;
\iota: G \to G\;\;\;\textrm{(inverse)}
\]
satisfying the axioms for composition and inverse.} In particular, 
since translates by elements of $g$ is a homeomorphism, the 
neighborhoods of the identity $e$ uniquely define the topology on 
$G$. Immediately, we have the following useful result:

\begin{lem}
Let $H$ be the intersection of all open neighhoods of $G$. Then
\begin{enumerate}
\item $H$ is a normal subgroup of $G$.

\item $H$ is the closure of $\{e\}$.

\item $G/H$ is Hausdorff.

\item $G$ is Hausdorff if and only if $H = \{e\}$.
\end{enumerate}
\end{lem}

A version of this proof for an abelian group $G$ can be found as
\cite{AM} Lemma 10.1.

\begin{proof}
\begin{enumerate}
\item fix $h \in H$, then $h^{-1} \in H$ (since taking inverse is 
also a homeomorphism). In particular, translates of any open 
neighborhood by $h$ contains $e$. It follows that $hH = H$, and 
that $H$ is a subgroup. Similarly, conjugating by $g \in G$ is a 
homeomorphism, whence $gHg^{-1} = H$. Hence $H \nsubg G$.

\item Fix $U$ an open set of $G$ that does not contain $e$. We
claim that $U$ does not meet $H$. Suppose not. Then fix $h \in U 
\cap H$. But $e \in hU^{-1}$, and thus $hU^{-1}$ must be a 
neighborhood of $e$. By definition, $h \in hU^{-1}$. It follows
that $e \in U$, a contradiction.

Therefore, $H$ is closed, and any closed set containing $e$
also contains $H$. Thus, $H$ is the closure of $\{e\}$.

\item By (1), $G/H$ is also a group, with topology given by the
quotient topology. The map $G \to G/H$ is an open surjection (and 
thereby also a closed map). Since $H$ is closed, the set $\{H\}$ 
in $G/H$ is closed. Let $G' \defeq G/H$, and let us abuse 
notations and write $e$ for the coset $H$.

In particular, $G' \times G' \to G'$ given by $(g, h) \mapsto 
g^{-1}h$ is continuous, and the inverse image of the closed set 
$\{e\}$ is precisely the diagonal, which is closed. It follows 
that $G/H$ is Hausdorff.

\item If $H = e$, then $G/H = G$, and therefore, $G$ is Hausdorff
by (3). Conversely, if $G$ is Hausdorf, then for any $h \notin G$,
there exists some open set $U$ containing $e$ but not $h$. It
follows that $H = e$, as desired.
\end{enumerate}
\end{proof}

In the case where the fundamental system of neighborhoods 
$\{G_\alpha\}$ of $e$ are given by normal subgroups of $G$, then 
we have an inverse system.

\begin{defn}
Let $G$ be given as above. The \DEF{group completion}{completion}
$\Complete{G}$ of $G$ is given by the inverse limit
\[
\Complete{G} = \varprojlim_{G_\alpha \subg G} G/G_\alpha.
\]
\end{defn}

\begin{rmk}
Here, we will not show why the group completion of such a 
topological group is a group or a topological space. Both 
statements are relatively straightforward to conceive, both 
statements are true and both are treated in Appendix 
\ref{appendix:profinite} in some terribly gruesome details. If
the reader should be so interested, please consider herself
duly warned, and placing her curiosity above efficiency.
\end{rmk}

There are two main classes of examples discussed in these notes.
The first will be treated quite extensively in this section, and 
forms the foundation for the rest of this section. The second will 
be the subject of study in Appendix \ref{appendix:profinite}. For
now, we make the following assumptions that will remain with us 
for the rest of the section:

\begin{enumerate}
\item $G$ is an abelian group,

\item the system of neighborhood form a countable tower of (normal)
subgroups
\[
G = G_0 \supseteq G_1 \supseteq G_2 \supseteq \cdots \supseteq G_n 
\supseteq \cdots.
\]
\end{enumerate}

In this case, the completion has yet another more ``analytic'' 
description.

\begin{thm}{Artin-Rees Lemma}\label{thm:artin_rees_lemma}
Let $R$ be a Noetherian ring, $\ideal{a}$ an $R$-ideal, and $M$ a
finitely generated $R$-module, and $F^*M$ a stable 
$\ideal{a}$-filtration of $M$. If $M'$ is a submodule of $M$ then
$(M' \cap F^nM)$ is a stable $\ideal{a}$-filtration of $M'$.
\end{thm}
