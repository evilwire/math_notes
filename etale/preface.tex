\Preface

First, the disclaimer. The author intends to do no harm to the
mathematical society as a whole by fillings its already ample
supply of texts on this subject with yet another text. A thick
one, no less! Second, a concession. When the author tried to
learn the very same material, the author had a very tough time.
For one, the material is scattered throughout the universe of
mathematics, and without necessarily a good reference book, it
was very painful to piece together the material from the great
expanse.

Then there came the stack project \cite{Stack}, which is (and 
remain to be) the great encyclopaedia of algebraic geometry 
culminating in the development of the theory of algebraic stacks. 
This more than 2,000 page tome formed a complete exposition of 
\'etale theory, but its catacombs of information became difficult 
to navigate. In these notes, the author had hoped desperately to 
form coherent ideas around the subject of \'etale cohomology and 
its applications that are as self-contained as possible. This 
\emph{may not} be possible --- something that became clear at the 
very beginning of this venture, but a gap can still be filled for 
those who wish to learn the subject having braved the first three 
chapters of Hartshorne's celebrated text (or its equivalent).

These notes came at the end of the author's graduate career,
as a distraction from his duties to his thesis. There are three
main goals to writing them. First and foremost, they are intended
to be as humorous as possible; so as to make the subject more
memorable so that there are now choice phrases peppered throughout 
these pages that can be repeated amidst the graduate ranks. 
Unfortunately, more genius went into coming up with these sentences
than the actual mathematics; the author, thereby, wishes to 
apologize for this obvious short-coming.

The second goal is to serve as a reference for a number of 
graduate students at the time who wish to learn the subject. This
is a modest goal, and in-so-far as this goal is concerned, it
is met.

We develop basic theories of \'etale cohomology theory. One can
view these notes as a verbose companion to Milne's ``\'Etale
Cohomology''.
