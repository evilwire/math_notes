\begin{prop}
Pushout of surjections are surjections.
\end{prop}

\begin{proof}
Let $f: A \to B$ be a surjection, and form the push-out $P$ of the 
following
\[
\begin{tikzcd}
A \arrow[twoheadrightarrow]{r}{f} \arrow{d}{g} &
B \\
C.
\end{tikzcd}
\]
Thus we have
\[
\begin{tikzcd}
0 \arrow{r} & 
K \arrow{r}{i}\arrow[dotted]{d} &
A \arrow{r}{f}\arrow{d}{g} &
B \arrow{r}\arrow{d}&
0 \\
0 \arrow{r} &
K' \arrow{r}{i'} &
C \arrow{r}{h} \arrow{rd}&
P \\
& & &\cok i' \arrow{r} &
0
\end{tikzcd}
\]
where $K$ is the kernel of $f$, and $i': K' \to C$ is the kernel 
of the push-out map $C \to P$. Since $hgi$ factors through $B$
and therefore $0$, there exists a map $K \to K'$ (dotted arrow
in the above diagram) such that $K \to K' \stackrel{i'}{\to} C$
equals $gi$. To show that $h$ is surjective, we show that it is 
the cokernel of $i'$.

Certainly, there exists a map $\cok i' \to P$, since $K' \to C 
\to P$ is 0. 

On the other hand, notice that $K \stackrel{i}{\to} A 
\stackrel{g}{\to} C \to \cok i'$ is 0, and there exists a map
$B \to \cok i'$ such that $A \to B \to \cok i'$ commutes with
$A \to C \to \cok i'$. But $P$ is the pushout. Hence, there
exists a map $P \to \cok i'$. 

As both of these maps are unique, it follows that the two maps
are inverses and are define isos.
\end{proof}
