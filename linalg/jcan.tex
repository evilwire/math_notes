\section{Jordan Canonical Form}

We want to prove the following statement: 

\begin{thm}[Existence of Jordan Canonical Form]
Let $M : V \to V$ be a linear map from an $n$-dimensional 
$\C$-vector space to itself. Then there exists a basis 
$\{e_1,\dots,e_n\}$ of $V$ such that the linear map $M$
can be represented by a block-diagonal matrix:
\[
\left(
\begin{matrix}
A_1 &0   & 0     &0 \\
0   &A_2 & 0     &0 \\
0   &0   &\ddots &0 \\
0   &0   &0      &A_k
\end{matrix}
\right)
\]
such that $A_i$ is a matrix of the form
\[
\left(
\begin{matrix}
\lambda_i & 1         & 0      & 0         & 0 \\
0         & \lambda_i & 1      & 0         & 0 \\
0         & 0         & \ddots & 1         & 0 \\
0         & 0         & 0      & \lambda_i & 1 \\
0         & 0         & 0      & 0         & \lambda_i
\end{matrix}
\right)
\]
for some $\lambda_i$ in $\C$. To clarify, the $\lambda_i$'s can 
take on any value. They do not have to be distinct; they can also
be 0.
\end{thm}

This is essentially the argument given in Artin's book (and is
univeral amongst algebra texts that present this result, though 
the style of the presentation may be different).

The main idea of the proof is to show that we can think of $V$
as a module over the polynomial ring $\C[x]$. Using the fact
that we know exactly the structure of these modules because
$\C[x]$ is a principle ideal domain (), 

we obtain
a characterization of

There are a number of results which we will mention in passing, 
and we will not prove them in detail for the sake of brevity, 
though please feel free to ask me to clarify as needed:

\begin{lem}\label{lem:Cx_is_PID}
Let $\C[x]$ be the polynomial algebra in one (free) variable. Then
$\C[x]$ is an Euclidean domain and a principle ideal domain (PID).
\end{lem}
\begin{proof}
The somewhat sketchy proof goes like this: an Euclidean domain
is PID (see Artin Chapter 11, Proposition 2.20 on page 398). It 
suffices, then, to show that $\C[x]$ is an Euclidean domain.

Recall that a domain is a ring such that the product of any two 
nonzero elements of the ring cannot be $0$. That is, $R$ is a
domain if for elements $f, g$ of $R$, $fg = 0$ then $f = 0$ or
$g = 0$. 

Showing that $\C[x]$ is a domain is straightforward.
\footnote{A proof: to see that $\C[x]$ is a domain, we need to 
show that the product of two nonzero polynomials cannot be zero. 
Pick two nonzero polynomials $f(x)$ and $g(x)$, and let $ax^n$ and
$bx^m$ be the leading term of $f(x)$ and $g(x)$ respectively. The 
product $f(x)g(x)$ has a leading term $abx^{n + m}$ which is 
certainly not zero.}

To see that $\C[x]$ has satisfy the Euclidean condition, we will
must show that for any $f(x)$ and $q(x)$ in $\C[x]$ and with
polynomial degree $n$ and $m$ respectively. Then there exists
$g(x)$ such that $f(x) = g(x)q(x) + r(x)$ where the polynomial
degree of $r(x)$ is less than $m$. This condition is modelled 
after long division with remainder. While is this not the most
general notion of a ring being Euclidean (see Artin Chapter 11,
2.17 on page 397), for all intents and purposes, this is good
enough for us.

Let $f(x)$ and $q(x)$ (which has polynomial degree $m$) be 
arbitrary polynomials in $\C[x]$. We proceed by induction on the 
difference between the degree of $f(x)$ and $q(x)$ (starting with 
\[
\deg f(x) - \deg q(x) = -1,
\] 
i.e. $\deg f(x) < \deg q(x)$). If $f(x)$ has a degree strictly 
less than $q(x)$ , then we are done: let $g(x) = 0$ and $r(x) = 
f(x)$, and the condition is satisfied. 

Now, suppose the Euclidean condition holds for all polynomials 
whose degree is $n$ more than that of $q(x)$, i.e. all polynomials
of degree $n + m$; let $f(x)$ be some polynomial whose degree is 
$n + m + 1$. Suppose further that $ax^{n + m + 1}$ is the leading 
term of $f(x)$ and $bx^m$ is the leading term of $q(x)$. Notice
that we can cancel the leading term of $f(x)$ by subtracting from
it a multiple of $q(x)$: let $h(x)$ be the polynomial $f(x) - 
\frac{a}{b}x^{n + 1}q(x)$. Then, $h(x)$ is a polynomial with 
degree $n + m$. By the induction hypothesis, $h(x) = g(x)q(x) +
r(x)$. So $f(x) = (g(x) + ax^{m + 1}/b)q(x) + r(x)$.
\end{proof}

\begin{lem}[The Structure Theorem of PIDs]
\end{lem}

This is essentially a generalization of the arguments provided
in Artin Chapter 12 Theorem 6.4 on page 472.

\begin{proof}

\end{proof}
